\chapter{Design}\label{k_design}
\section{Systemüberblick}
Die Systemkomponenten wurden in drei Gruppen eingeteilt:
\begin{itemize}
	\item Gruppe \textbf{Hardware}: Die Komponenten dieser Gruppe stellen eine direkte Abstraktion der Sensoren und Aktoren dar, mit denen direkt interagiert werden kann.
	Dies umfasst die LEDs, den Photo- und den Hallsensor, den Trigger und das Release.
	\item Gruppe \textbf{Model}: In dieser Gruppe befinden sich (physikalisch vorhandene) Objekte, mit denen nicht direkt interagiert werden kann, deren Modellierung als eigenes Objekt dennoch den Systementwurf vereinfacht.
	In der derzeitigen Systemgestaltung befindet sich in dieser Gruppe nur die Scheibe.
	\item Gruppe \textbf{Controller}\footnote{Die Ähnlichkeit der Gruppenamen zum MVC-Pattern ist zufällig entstanden und wurde nicht beabsichtigt.}:
	Die Komponenten in dieser Gruppe implementieren die Ablauflogik des Programms.
	Durch ein einheitliches Interface dieser Komponenten ist einerseits ein einfacher Wechsel zwischen verschiedenen Programmfunktionen möglich, andererseits bleiben unterschiedliche Funktionen getrennt.
	Hier wurde die Hauptaufgabe (automatischer Kugelfall), ein manuell gesteuerter Abwurf und die Ausgabe der Sensorwerte im CSV-Format als jeweils eigene Komponente realisiert.
\end{itemize}

\section{Vorgehen}
TODO

\section{Grenzwerte}
TODO

\section{Fehlerbetrachtung}
TODO