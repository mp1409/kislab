\chapter{Zusammenfassung} \label{k_zusammenfassung}
Wie aus den Messungen hervorgeht wurden die Anforderungen für das System unter den in der Einleitung genannten Einschränkungen erfüllt.
Die Messungen wurden per Hand durchgeführt da eine automatische Erkennung ob die Kugel das loch getroffen hat nicht vorgesehen ist.
Des weiteren gibt es durch Hardware ungenauigkeiten Fälle in denen die Kugel seitlich neben dem Loch auf der Scheibe aufprallt und wegfliegt, ebenso wie Situationen in denen sie an der innenkante abprallt und aus dem Loch wieder rausspringt.
Diese Fälle konnten durch einen bereitgestellte Webcam im Zeitraffer untersucht werden.
In unseren Messungen haben wir diese Situationen als gültig verbucht, es waren ungefährt $5$\% der versuchsdurchführungen die sich so verhalten haben.

Eine verbesserung des Systems könnte noch an der Bremserkennung vorgenommen werden, welche bis jetzt noch nicht in allen Fällen zuverlässig greift.
Drehrichtung und den Versuchsaufbau konnten hingegen frei gewählt werden und konnten mit minimalen korreckturen die gleichen Ergebnisse erzielen.