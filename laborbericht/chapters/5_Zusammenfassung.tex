\chapter{Zusammenfassung} \label{k_zusammenfassung}
Wie aus den Messungen hervorgeht wurden die Anforderungen für das System unter den in der Einleitung genannten Einschränkungen erfüllt.
Die Messungen wurden per Hand durchgeführt da eine automatische Erkennung ob die Kugel das Loch getroffen hat nicht vorgesehen ist.
Des weiteren gibt es bedingt durch Ungenauigkeiten in der Hardware Fälle in denen die Kugel seitlich neben dem Loch auf der Scheibe aufprallt, ebenso wie Situationen in denen sie an der Innenkante der Öffnung abprallt und wieder herausspringt.
Diese Fälle konnten durch die Kamera eines bereitgestellten Debuggers in Zeitlupe untersucht werden.
In unseren Messungen haben wir diese Situationen (betraf etwa 5 \% der Fälle) als gültig verbucht.

Eine Verbesserung des Systems könnte noch an der Erkennung von Unregelmäßigkeiten in der Drehbewegung (\abk{z}{B} manuelles Bremsen) vorgenommen werden, welche bis jetzt noch nicht in allen Fällen zuverlässig greift.
Drehrichtung und den Versuchsaufbau konnten hingegen frei gewählt werden und vergleichbare Ergebnisse erzielen.
