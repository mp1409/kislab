\documentclass[%fontsize=12pt,
paper=a4,%Papierformat
parskip=half,%Abstand statt Einzug für Absatzwechsel
headsepline,%Linie unter Kopfzeile
plainheadsepline,%Linie unter Kopfzeile bei Kapitelanfängen
twoside=false,%Einseitiger Satz
headings=small%Kleine Kapitelüberschriften
]{scrreprt}

%%%PAKETE	
%%Grundlegende Pakete
\usepackage[utf8]{inputenc}%Eingabezeichensatz
\usepackage[T1]{fontenc}%Verwende Type 1-Schriften
\usepackage{lmodern}%Schriftart
\usepackage[ngerman]{babel}%Deutsche Silbentrennung
\usepackage{microtype}

%% Keine Warnings bei float
\usepackage{scrhack}

%%Pakete für Zeichen
\usepackage{textcomp}%Symbole
\usepackage{latexsym}%Symbole
\usepackage[style=german]{csquotes}%Anführungszeichen mit \enquote{}

%%Pakete für Schriften
\usepackage{color}

%%Pakete für Formatierung
\usepackage{scrpage2}%Paket für KOMA-Script Kopf- und Fußzeilen
\usepackage{geometry}%Seitenränder anpassen

%%Pakete für Tabellen
\usepackage{multirow}
\usepackage{booktabs}

%%Pakete für Mathematik
\usepackage{amsmath}%Mathematikpakte der AMS
\usepackage{amsfonts}%Schriften der AMS
\usepackage{amssymb}%Symbole der AMS
\usepackage{nicefrac} %Schräger Bruch

%%Pakete für Grafiken
\usepackage{graphicx}

%%Pakete für Quelltext
\usepackage{listings}

%%Sonstige Pakete
\usepackage[iso]{datetime}
\usepackage{ifthen}
\usepackage{caption}
\usepackage{subcaption}

%%%BEFEHLE
%%Befehl für typographisch korrekten Abstand bei Abkürzungen wie "z. B." oder "d. h."
\newcommand{\abk}[2]{{#1}.\,{#2}.}
\newcommand{\identifier}[1]{\texttt{#1}}

%%Sonstige Befehle und Variablen
\newboolean{draft}

%%%EINSTELLUNGEN
%%Grundlegende Einstellungen für das Dokument
\setboolean{draft}{true}

\title{Laborbericht}
\subtitle{Komplexe Informationstechnische Systeme - Grundlagen}
\author{Lennard Pfennig \and Simon Buttgereit \and Michael Pfeiffer}
\date{6. Juli 2016}
\subject{Sommersemester 2016}

% Wenn Boolean draft gesetzt ist, setze Hinweis auf Titelseite
\ifthenelse{\boolean{draft}}{%if then
	\titlehead{\centering{\textcolor{red}{
		{\Huge Entwurf}\\
		Kompiliert: \today \ \currenttime
	}}}
}{}

%%Schriftart
\renewcommand{\familydefault}{\sfdefault}%Serifenlose Schrift

%%Seitenrandeinstellungen
\geometry{a4paper,left=25mm,right=25mm, top=30mm, bottom=30mm}

%%Anpassungen von Kapitelüberschriften
\renewcommand*{\chapterheadstartvskip}{\vspace{-5mm}}

%%Anpassen von Kopf- Fußzeilen
\pagestyle{scrheadings}
\automark{chapter}
\ohead[\headmark]{\headmark}
\chead{}
% Wenn Boolean draft gesetzt ist, setze Hinweis auf jeder Seite oben links
\ifthenelse{\boolean{draft}}{%if then
	\ihead[\textnormal{\textcolor{red}{Entwurf - Kompiliert: \today \ \currenttime}}]
	{\textnormal{\textcolor{red}{Entwurf - Kompiliert: \today \ \currenttime}}}
}
{%else
	\ihead{}
}

%%Einstellungen für Quelltext
\lstset{ %
language=c++,%Standardsprache
basicstyle=\footnotesize\ttfamily,
breaklines=true,
%showstringspaces=false,
flexiblecolumns=true,
numbers=right,
numberstyle={\tiny}
}

%%Einstellungen für die PDF-Erzeugung
%Hyperref muss als letztes aufgerufen werden!
\usepackage[pdfstartview = FitH,%Seiten auf volle Breite anzeigen
bookmarksopen=true,bookmarksnumbered=true,%Inhaltsverzeichnis im PDF links anzeigen
colorlinks,linkcolor=black,plainpages=false,hypertexnames=false,citecolor=black,filecolor=black,urlcolor=black,%Keine Hervorhebung von Links
pdfpagelabels,pdftitle={Laborbericht},
pdfauthor={Lennard Pfennig, Simon Buttgereit und Michael Pfeiffer},
pdfsubject={Komplexe Informationstechnische Systeme}]{hyperref}%PDF-Informationen
\usepackage[ngerman]{cleveref}

\pdfcompresslevel=0%PDF-Kompression (Bilder)

\begin{document}
\maketitle

\tableofcontents

\chapter{Einleitung}
\section{Abschnitt}
bla bla

\chapter{Analyse}\label{k_analyse}

% Beschreibung des Aufbaus
\section{Versuchsaufbau}

Der Versuchsaufbau besteht aus einer drehbaren Scheibe mit Loch und einer Kugelabwurfvorrichtung.
Diese wird mit einem Servo bedient (siehe Abschnitt \ref{subs:aktoren}).
Abbildung \ref{img:versuchsaufbau} zeigt eine schematische Darstellung.
Da das Ziel des Praktikums ist, Kugeln durch das Loch fallen zu lassen, ist die Abwurfvorrichtung so angebracht, dass die Kugeln bei einer bestimmten Position der Scheibe durch dieses fallengelassen werden können.
Die Unterseite des Scheibe besitzt zwei Farbschattierungen (hell und dunkel), welche durch einen Photosensor erkannt werden können.
Weiterhin sind zwei Magneten angebracht, welche durch einen Hallsensor erkannt werden.
Im folgenden wird genauer auf die Sensoren eingegangen.



\begin{figure}[h] \centering
	\includegraphics[width=\textwidth]{images/aufbau.pdf}
	\caption{schmatischer Versuchsaufbau - v.\,l.\,n.\,r. Seitenansicht Gesamtaufbau, Drehbare Scheibe}
	\label{img:versuchsaufbau}
\end{figure}

\begin{figure}[h] \centering
\begin{tabular}{lc} 
	\textbf{Eigenschaft} 	& \textbf{Beschreibung}	\\
	\toprule
	\multicolumn{2}{c}{Tisch}\\ 
	\midrule
	Abwürfhöhe 	& 750\,mm \\
	\#Schwarzer Felder 	& 6 \\
	\#Weiße Felder 	& 6 \\
	Lochlänge innen 	& 55\,mm \\
	\midrule 
	\multicolumn{2}{c}{Kugel}\\ 
	\midrule
	Masse 	& 8.5\,g \\
	Durchmesser 	& 12\,mm \\
	\bottomrule
\end{tabular}
\end{figure}

\subsection{Sensoren}
Der Hallsensor zeigt zwei mal pro Umdrehung exakt die Position der Scheibe an.
Dies geschieht, wenn das Loch an \texttt{HallSensorPosition1} und \texttt{HallSensorPosition0} ist (siehe Abb. \ref{img:versuchsaufbau}), wobei zweiteres der Position entspricht, an welcher die Kugel durch das Loch fallen muss.
An den Positionen wechselt die Ausgabe der Sensoren auf den entsprechenden Wert, daher 1 bzw. 0.

Der Photosensor gibt abhängig davon, ob er eine helle oder dunkle Fläche vor sich hat eine 1 oder 0 aus.
Da die Felder auf der Unterseite der Scheibe gleichmäßig verteilt sind (siehe Abb. \ref{img:versuchsaufbau}), kann er nicht zur Positionsbestimmung, dafür für die Bestimmung der Drehgeschwindigkeit verwendet werden.

Die Sensorwerte einer Scheibenumdrehung sind in Abb. \ref{img:sensorwerte} geplottet.

\textbf{TODO: Trigger}

\subsection{Aktoren}
\label{subs:aktoren}
Wie bereits beschrieben, wird die Abwurfvorrichtung durch einen Servo gesteuert.
Sobald sich dieser aus seiner Ursprungsposition auf einen Winkel von 30° dreht, fällt eine Kugel aus dem Magazin und die nächste Kugel wird automatisch nachgeladen.

Weiterhin gibt es noch einige LEDs.

\subsection{Ansteuerung der Komponenten}
Für die Ansteuerung stehen ein Arduino UNO und eine Art Blackbox zur Verfügung.
Zweitere sorgt dafür, dass die Sensoren und Aktoren an die Pins des Arduino angeschlossen sind, und einfach angesteuert werden können.
Die Belegung der Pins sind in folgender Tabelle beschrieben.


\begin{figure}[h] \centering
\begin{tabular}{cll} 
	\textbf{Pin} 	& \textbf{In/Out} & \textbf{Funktion}	\\
	\toprule
	2 &	Input &	Photosensor \\
	3 &	Input &	Hallsensor\\
	4 &	Input &	Trigger\\
	5 &	Input &	Switch\\
	7 &	Output &	Blackbox LED gelb\\
	9 &	Output &	Servo\\
	10 &	Input &	Button 1\\
	11 &	Input &	Button 2\\
	12 &	Output &	LED 1\\
	13 &	Output &	LED 2\\
	\bottomrule
\end{tabular}
\end{figure}

\begin{figure}[h] \centering
	\includegraphics[width=\textwidth]{images/generated/sensor_messwerte1.pdf}
	\caption{Sensorzyklus einer Scheibenumdrehung}
	\label{img:sensorwerte}
\end{figure}

\section{Fallzeit der Kugel}
Um die Fallzeit einer Kugel zu berechnen wird die Formel für den freien Fall im homogenen Feld
\begin{align}
	t(h) = \sqrt{\frac{2h}{g}}
\end{align}
verwendet.
Setzt die oben beschriebenen, am Versuchsaufbau gemessenen, Messwerte ein

\begin{align}
t(h) &= \sqrt{\frac{2 \cdot 0.75m}{9.81\,\nicefrac{m}{s^2}}} \\
	 &= 0.391\,s \\
	 &= 391\,ms
\end{align}
so kommt man auf eine Fallzeit von 391\,ms für eine Kugel.



% Approximationskurve
\begin{figure}[hb] \centering
	\includegraphics[width=\textwidth]{images/generated/Data4.pdf}
	\caption{Drehgeschwindigkeit über der Zeit approximiert mit der Funktion: \newline $u(t) = 150.9575 - 1.5678\cdot t + 0.0036 \cdot t^2$}
\end{figure}


\chapter{Design}\label{k_design}

% Systemüberblick (3 Komponenten)

% Vorgehen

% Grenzwerte

% Fehlerbetrachtung
\chapter{Implementierung}

% Übersicht Komponenten -> Klassen

% Kontrollfluss

% Validierung
\chapter{Zusammenfassung}
Blablabla war das toll

\end{document}
